\documentclass[]{article}
\usepackage{lmodern}
\usepackage{amssymb,amsmath}
\usepackage{ifxetex,ifluatex}
\usepackage{fixltx2e} % provides \textsubscript
\ifnum 0\ifxetex 1\fi\ifluatex 1\fi=0 % if pdftex
  \usepackage[T1]{fontenc}
  \usepackage[utf8]{inputenc}
\else % if luatex or xelatex
  \ifxetex
    \usepackage{mathspec}
  \else
    \usepackage{fontspec}
  \fi
  \defaultfontfeatures{Ligatures=TeX,Scale=MatchLowercase}
\fi
% use upquote if available, for straight quotes in verbatim environments
\IfFileExists{upquote.sty}{\usepackage{upquote}}{}
% use microtype if available
\IfFileExists{microtype.sty}{%
\usepackage{microtype}
\UseMicrotypeSet[protrusion]{basicmath} % disable protrusion for tt fonts
}{}
\usepackage[margin=1in]{geometry}
\usepackage{hyperref}
\hypersetup{unicode=true,
            pdftitle={Metodología de la investigación filosófica},
            pdfborder={0 0 0},
            breaklinks=true}
\urlstyle{same}  % don't use monospace font for urls
\usepackage{longtable,booktabs}
\usepackage{graphicx,grffile}
\makeatletter
\def\maxwidth{\ifdim\Gin@nat@width>\linewidth\linewidth\else\Gin@nat@width\fi}
\def\maxheight{\ifdim\Gin@nat@height>\textheight\textheight\else\Gin@nat@height\fi}
\makeatother
% Scale images if necessary, so that they will not overflow the page
% margins by default, and it is still possible to overwrite the defaults
% using explicit options in \includegraphics[width, height, ...]{}
\setkeys{Gin}{width=\maxwidth,height=\maxheight,keepaspectratio}
\IfFileExists{parskip.sty}{%
\usepackage{parskip}
}{% else
\setlength{\parindent}{0pt}
\setlength{\parskip}{6pt plus 2pt minus 1pt}
}
\setlength{\emergencystretch}{3em}  % prevent overfull lines
\providecommand{\tightlist}{%
  \setlength{\itemsep}{0pt}\setlength{\parskip}{0pt}}
\setcounter{secnumdepth}{0}
% Redefines (sub)paragraphs to behave more like sections
\ifx\paragraph\undefined\else
\let\oldparagraph\paragraph
\renewcommand{\paragraph}[1]{\oldparagraph{#1}\mbox{}}
\fi
\ifx\subparagraph\undefined\else
\let\oldsubparagraph\subparagraph
\renewcommand{\subparagraph}[1]{\oldsubparagraph{#1}\mbox{}}
\fi

%%% Use protect on footnotes to avoid problems with footnotes in titles
\let\rmarkdownfootnote\footnote%
\def\footnote{\protect\rmarkdownfootnote}

%%% Change title format to be more compact
\usepackage{titling}

% Create subtitle command for use in maketitle
\newcommand{\subtitle}[1]{
  \posttitle{
    \begin{center}\large#1\end{center}
    }
}

\setlength{\droptitle}{-2em}

  \title{Metodología de la investigación filosófica}
    \pretitle{\vspace{\droptitle}\centering\huge}
  \posttitle{\par}
    \author{}
    \preauthor{}\postauthor{}
    \date{}
    \predate{}\postdate{}
  
\usepackage{fontspec}
\setmainfont{Adobe Jenson Pro}

\begin{document}
\maketitle

\section{Descripción del curso}\label{descripcion-del-curso}

En la monografía de grado, el estudiante debe articular los
conocimientos adquiridos y las competencias desarrolladas en las
asignaturas y seminarios que componen la carrera. Este trabajo de grado
consiste en una monografía que estudie un problema filosófico en un
autor o corriente, utilizando las metodologías propias de la
investigación en filosofía. Debe evidenciarse un conocimiento suficiente
de la tradición filosófica relevante, así como un adecuado uso de las
fuentes, de la bibliografía secundaria y de los conceptos pertinentes.

El objetivo central de este curso es el diseño de un anteproyecto que
sirva para estructurar su trabajo grado. Para ello, se trabajará en
pulir y complementar las herramientas obtenidas a lo largo de la
carrera, perfilándolas hacia la elaboración de una monografía. Al
finalizar este curso el estudiante deberá obtener la aprobación del
proyecto de tesis de el profesor o profesora que vaya a dirigir el
trabajo presentándo un anteproyecto.

\textbf{Docente}: \href{../index.html}{Juan Camilo Espejo-Serna}\\
\textbf{Horario y salón:} Lunes 8:00 - 11:00 am, Atelier 103\\
\textbf{Página web del curso}: \url{http://jcunisabana.github.io/MIF/}

\section{Objetivos}\label{objetivos}

\begin{enumerate}
\def\labelenumi{\arabic{enumi}.}
\item
  Identificar un tema de tesis apropiado para el proyecto de
  investigación.
\item
  Escribir una primera síntesis del problema filosófico a tratar.
\item
  Escribir una primera presentación del estado del arte en el tema.
\item
  Formular una hipótesis de investigación preliminar.
\item
  Identificar y caracterizar los objetivos de la investigación.
\item
  Elaborar un primer cronograma de trabajo para terminar la monografía.
\item
  Realizar una selección de bibliografía relevante para el trabajo de
  grado.
\item
  Estar en posición de ofrecer una justificación de la pertinencia de la
  investigación desde sus antecedentes.
\item
  Usar adecuadamente el vocabulario y registro propio de la escritura
  académica.
\item
  Emplear la gramática, la puntuación y la ortografía adecuada para
  construir un texto coherente.
\item
  Utilizar apropiadamente uno de los diferentes estilos de citación
  reconocidos en la comunidad académica (APA, Chicago, MLA, etc.).
\end{enumerate}

\section{Contenido temático}\label{contenido-tematico}

\emph{Módulo 1: Identificación de un tema}

Delimitación de un tema. Creación de una primera bibliografía. Lectura
estructurada. Retroalimentación entre pares.

\emph{(5 semanas)}

\emph{Módulo 2 - Delimitación de un tema}

Lectura estructurada. Formulación de una hipótesis de investigación.
Primera síntesis del asunto a tratar. Búsqueda de directores de tesis.
Primera versión del anteproyecto

\emph{(6 semanas)}

\emph{Módulo 3 - Aprobación del anteproyecto}

Lectura estructurada. Primeras reuniones con los de directores de tesis.
Segunda versión del anteproyecto.

(\emph{6 semanas})

\section{Calificación}\label{calificacion}

\begin{longtable}[]{@{}lcc@{}}
\toprule
\emph{Actividad} & \emph{Valor porcentual} & \emph{Corte}\tabularnewline
\midrule
\endhead
Entrega semanal & 15\% & 1\tabularnewline
Participación & 15\% & 1\tabularnewline
Entrega semanal & 15\% & 2\tabularnewline
Participación & 15\% & 2\tabularnewline
Anteproyecto & 40\% & 3\tabularnewline
\bottomrule
\end{longtable}

\paragraph{Entrega semanal}\label{entrega-semanal}

Para cada semana, los estudiantes deben leer dos artículos o un capítulo
de un libro relevante y producir las reseñas correspondientes según las
indicaciones del profesor. La entrega deberá ser subida a la plataforma
virtual a más tardar el \emph{domingo a las 7 de la noche}. Todos los
estudiantes empiezan con 5.0 en esta nota; por cada vez que no se
participe dentro del rango de tiempo especificado, la nota será
disminuida de acuerdo con los siguientes parámetros: primera vez: -0.5;
segunda vez: -1.0; tercera vez: -1.5; cuarta vez: -2.0.

\paragraph{Participación}\label{participacion}

Cada semana, se deberá comentar al menos una entrega de sus compañeros.
Hagan una pregunta, sugieran bibliografía, propongan un cambio, etc.. Es
importante que los comentarios sean articulados y busquen mejorar el
texto de sus compañeros. Los comentarios deben ser realizados en las
\emph{24 horas siguientes a la clase}. Todos los estudiantes empiezan
con 5.0 en esta nota; por cada vez que no se participe dentro del rango
de tiempo especificado, la nota será disminuida de acuerdo con los
siguientes parámetros: primera vez: -0.5; segunda vez: -1.0; tercera
vez: -1.5; cuarta vez: -2.0.

\paragraph{Borrador anteproyecto}\label{borrador-anteproyecto}

Primera versión del anteproyecto siguiendo el formato dado por el
profesor.

\paragraph{Anteproyecto}\label{anteproyecto}

Segunda versión del anteproyecto siguiendo el formato dado por el
profesor. Debe incluir una defensa oral del proyecto ante compañeros y
quien puede ser el posible director o directora.


\end{document}
